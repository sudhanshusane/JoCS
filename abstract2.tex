We conduct an empirical study evaluating 
\textit{in situ} reduction of time-varying vector field data
from a Lagrangian perspective,
%in order 
to enable \textit{post hoc} exploration via flow visualization.
%
This paradigm has the potential to provide significant accuracy and storage
advantages over traditional techniques when storage is limited.
%
However, no prior study has demonstrated 
\textit{in situ} viability.
%the paradigm can be practically 
%achieved.
%
Our study considers state-of-the-art approaches and a variety
of workloads, varying over number of particles, interval between
outputting data, grid size, and concurrency.
%
It also considers evaluation criteria that span from \textit{in situ}
encumbrance to data storage costs to \textit{post hoc} accuracy
and performance.
%
The study incorporates integrations with three computational simulations,
with 47 experiments on a current top supercomputer.
%
%In all, our contribution is to show that the Lagrangian-based reduction
In all, our contribution shows that Lagrangian-based reduction
should be the preferred solution for an important
class of problems: exploratory time-varying large vector field visualization in I/O constrained settings. 
%when I/O is constrained.
